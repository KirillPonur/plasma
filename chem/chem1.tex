\documentclass[border=1pt]{standalone}
\usepackage[europeanresistors,americancurrents]{circuitikz}
\usepackage{amssymb}
% \usetikzlibrary{tikzmark,decorations.pathmorphing}
\begin{document}
    \begin{circuitikz}[]
        \draw (0,0) coordinate (0)

        to ++(-2,0) coordinate(1)
        to [L=$L_1$]++(0,-2) coordinate(2)
        % to [L=$L_1$]++(0,-2) coordinate(3)
        to ++(2,0)
        to [C=$C_1$,name=epsilon]++(0,2)
        to (0);
          \fill[fill=cyan!40]
          ([yshift=-\pgflinewidth]epsilon.se) 
            rectangle 
          ([yshift=\pgflinewidth]epsilon.nw);   
  
        \node at (epsilon.center)  
        {$\scriptstyle\varepsilon(\omega)$};


        \draw (-4,-1.1)
        to [short,o-]++(1,0)
        to [L,l_=$L_3$]++(0,-2) 
        to [short,-o]++(-1,0);
        % \draw                       [decorate,decoration=
                                    % {snake,segment length=6pt,amplitude=1pt}]
        % (1)+(-0.5,-0.5) to ++(-0.5,-1.5) node[left]{$M_{20}$};
        % \draw (1) node[](M)
        % \draw (3)+(-0.5,0.5) to ++(-0.5,1.5)
                                    % [decorate,decoration=
                                    % {snake,segment length=6pt,amplitude=1pt}]
                                    % node[left]{$M_{10}$};
        \draw (-4,1.1)
        to [short,o-]++(1,0)
        to [L,l_=$L_2$]++(0,-2)
        to [short,-o]++(-1,0);

    \end{circuitikz}
\end{document}