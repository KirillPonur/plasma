\documentclass[10pt,pdf,hyperref={unicode}, dvipsnames]{beamer}
\usepackage[english,russian]{babel}
% \usepackage[T2A,T1]{fontenc}
\usepackage[utf8]{inputenc}
\usepackage{tikz}
\usepackage[unicode]{hyperref}
\usepackage{pgfplots,standalone}
\usepackage[normalem]{ulem}
\usepackage
	{
		% Дополнения Американского математического общества (AMS)
		amssymb,
		amsfonts,
		amsmath,
		amsthm,
		physics
		}
% \usepackage{lmodern}
\pgfplotsset{compat=newest} 
\usetikzlibrary{%
    decorations.pathreplacing,%
    decorations.pathmorphing,%
    patterns,%
    angles,%
    quotes,%
    calc, %
    3d, %
    backgrounds, %
    positioning%
}


% Стиль презентации

 \usetheme{Warsaw}
 \usefonttheme{professionalfonts}
 \usecolortheme{}
 % \usecolortheme{whale}
% \let\oldframe\enumerate
% \renewcommand{\frame}{%
% \oldframe
% \let\olditemize\itemize
% \renewcommand\itemize{\olditemize\addtolength{\itemsep}{100pt}}%
% }
 

% \setbeamercolor{frametitle right}{fg=white,bg=Brown!85}
% \setbeamercolor{frametitle}{fg=white,bg=Brown!85}
%\setbeamercolor{frametitle right}{fg=white,bg=black!85} %
\setbeamercolor{frametitle}{fg=white,bg=black!85} % Цвет титульника

\setbeamertemplate{headline}{}
\setbeamertemplate{footline}{}
\let\Tiny=\tiny % решает проблему со шрифтами в TexLive
\setbeamertemplate
	{footline}{
		\color{black!40!white}
		\quad\hfill
		\insertframenumber/\inserttotalframenumber
		\hfill\vspace{1cm}\quad
	} 

\setbeamertemplate{navigation symbols}{}

\beamersetrightmargin{1cm} 
\beamersetleftmargin{1cm}

\setbeamertemplate{enumerate item}{
	\usebeamercolor[bg]{item projected}
	\raisebox{1pt}{\colorbox{bg}{\color{fg}\footnotesize\bf\insertenumlabel}}%
}
\setbeamercolor{item projected}{bg=black,fg=white}


\setbeamertemplate{itemize item}{%
	\usebeamercolor[bg]{item projected}%
	\raisebox{1pt}{{\color{bg}\footnotesize$\bf\square$}}%
}
\setbeamercolor{item projected}{bg=black,fg=white}
\setbeamercolor{title}{bg=black,fg=white}

\title[Измерение плотности плазмы]{Измерение плотности плазмы}

\author{%
	Виноградов И.Д. %
	Понур К.А. %
	Шиков А.П. %
}

\institute{Радиофизический факультет ННГУ, 430 группа}

\date{Нижний Новгород, 2018}

\begin{document}  

\section{Введение}
\subsection{Цели работы}
\begin{frame}[t]
	\frametitle{Цели работы}
	% \textbf{Цели}\\
		\vfill
		\begin{enumerate}
			\item Изучить принцип работы зонда с СВЧ-резонатором

			\item Измерить локальную концентрацию плазмы на экспериментальной установке "КРОТ"

			\item Изучить временную зависимость концентрации для распадающейся плазмы  

		\end{enumerate}
		\vfill
\end{frame}
\subsection{Актуальность работы}
\begin{frame}[t]
	
	\frametitle{Актуальность работы}
		
		\begin{enumerate}
			\item Моделирование космических явлений
			\item Моделирование процессов в ионосфере
			\item Идея управляемого термоядерного синтеза
			\item МГД-преобразование энергии и ионные двигатели
			\item Газовые лазеры
			\item Газоразрядные электронные приборы 
		\end{enumerate}


\end{frame}

\section{Краткая теория}
\subsection{Некоторые свойства плазмы}
\begin{frame}
	\frametitle{Некоторые свойства плазмы}
	\textbf{Плазма} -- частично или полностью ионизированный газ, образованный из нейтральных атомов (или молекул) и заряженных частиц.

	$\,$
	\begin{enumerate}
		\item Квазинейтральный газ
		\item Экранирует действующие на неё на неё электрические поля
		\item Высокая проводимость
		\item 
	\end{enumerate}
	Плазма является электрически нейтральной системой.
	% Особенность плазмы в том, что она экранирует действующие на неё электрические поля. 
\end{frame}

\subsection{Генерация плазмы}
\begin{frame}
	\frametitle{Генерация плазмы в КРОТе}
	Для генерации плазмы используется индукционный газовый разряд, осуществляемый при помощи высокочастотных вихревых электрических
	полей, создаваемых мощными катушками индуктивности, расположенными внутри установки.
	\begin{figure}[H]
		\begin{minipage}{\linewidth}
				\centering
				\includegraphics[width=0.4\linewidth]{fig/induct}
				\caption{Генерация ВЧ поля}
				\label{fig:resonator}
		\end{minipage}
		\end{figure}

\end{frame}



% subsection  (end)
% \section{Теоретическая часть} % (fold)
% \label{sec:теоретическая_часть}

% \subsection{Понятие плазмы} % (fold)
% \label{sub:понятие_плазмы}
% % \begin{frame}
		
% % \end{frame}
% \subsection{Источник плазмы} % (fold)
% \label{sub:генерация_плазмы}
% % \begin{frame}
	
% % \end{frame}

\subsection{Зонд с СВЧ-резонатором} % (fold)

\begin{frame}
	\frametitle{Зонд с СВЧ-резонатором}
	Идея метода замера локальной  плотности плазмы заключается в измерении собственной частоты резонатора, помещенного в неё.

	\begin{equation}
		\omega_{res}=\left(\frac\pi{2l}\right)\frac c{\sqrt{\varepsilon}},
	\end{equation}
	где $\omega_{res}$-- собственная частота резонатора.

	В плазме: $$\varepsilon=\varepsilon(\omega)=1-\frac{\omega^2}{\omega_p^2}.$$ 
	Тогда сдвиг резонансной частоты по сравнению с вакуумом:
	$$\omega_{res}^2=\omega_{0res}^2+\omega_p^2,$$ 
	где $\omega_{0res}$-- собственная частота резонатора в вакууме,а $\omega_p$-- плазменная частота

	При этом концентрация однозначно связана с плазменной частотой:
	\begin{equation}
		N_e=\frac{m_e\omega^2_p}{4\pi e^2}
	\end{equation}
	% Таким образом мы сможем найти концентрацию.
\end{frame}

\begin{frame}
	\frametitle{Зонд с СВЧ-резонатором}
	В нашем случае резонатором является четвертьволновый отрезок двупроводной линии (четвертьволновый резонатор), замкнутый на одном конце и разомкнутый на другом. 
	\begin{figure}[H]
	\begin{minipage}{0.49\linewidth}
			\centering
			\includegraphics[width=\linewidth]{fig/resonator}
			\caption{Четвертьволновый резонатор}
			\label{fig:resonator}
	\end{minipage}
	\begin{minipage}{0.49\linewidth}		
			\centering
			\includegraphics[]{chem/chem1}
			\caption{Эквивалентная съема резонатора}
			\label{fig:chem1}
	\end{minipage}
	\end{figure}
\end{frame}

\begin{frame}
	\frametitle{Преимущества зонда с СВЧ-резонатором}
	\begin{enumerate}
		\item Благодаря малым размерам резонатора, позволяет определять локальное, слабо возмущенное значение концентрации плазмы.
		\item Результаты измерений не зависят от температуры плазмы.
	\end{enumerate}
\end{frame}
\section{Эксперимент}
\subsection{Описание экспериментальной установки}
\begin{frame}
	\frametitle{Экспериментальная установка КРОТ}
	Размеры камеры: диаметр 3 м, длина 10 м.

	Размеры соленоида: длина 3.5 м, диаметр 2 м. 
	
	Магнитное поле достигает величины $B\approx 1000$ Эрстед. 
	
	Предельный вакуум, достигаемый в объеме камеры $P =3\cdot10^{-6}$ Торр. 
	\begin{figure}[tb]
		\vspace{0pt}
		\centering
		\includegraphics[width=0.6\linewidth]{fig/krot}
		\label{fig:krot}
		\caption{1 - вакуумная откачка, 2 - зонд с СВЧ-резонатором,3 - ВЧ-генератор, 4 - соленоид с источником магнитного поля.}
	\end{figure}
\end{frame}
\section{Практическая часть} % (fold)
\subsection{Измерение концентрации плазмы}
\begin{frame}
	\frametitle{Измерение концентрации плазмы}
	Рабочий газ -- аргон (Ag), 
	$P=4\cdot10^{-1}$ Торр. 
	\begin{figure}[tb]
		\vspace{-5pt}
		\centering
		\includegraphics[width=0.89\linewidth]{fig/concentration}
		\label{fig:1}
	\end{figure}
\end{frame}
\subsection{Распад  плазмы}
\begin{frame}
	\frametitle{Распад плазмы}
	\begin{figure}[tb]
		\centering
		\includegraphics[width=0.9\linewidth]{fig/decay}
		\label{fig:2}
	\end{figure}
\end{frame}

\subsection{Зависимость на радиальной оси}
\begin{frame}
	\frametitle{Зависимость на радиальной оси}
	Время после отключения ВЧ-генератора $t=4.13$ мс
	\begin{figure}[tb]
		\centering
		\includegraphics[width=0.9\linewidth]{fig/radial}
		\label{fig:2}
	\end{figure}
\end{frame}

\subsection{Выводы}
\begin{frame}
		\vfill
		\frametitle{Выводы}
		\begin{enumerate}
			\item Мы изучили принцип работы зонда с СВЧ-резонатором

			\item Измерили локальную концентрацию плазмы на экспериментальной установке "КРОТ"

			\item Изучили временную зависимость концентрации для распадающейся плазмы  

		\end{enumerate}
		\vfill
\end{frame}




\end{document}