\documentclass[10pt,pdf,hyperref={unicode}, dvipsnames]{beamer}
\usepackage[english,russian]{babel}
% \usepackage[T2A,T1]{fontenc}
\usepackage[utf8]{inputenc}
\usepackage{tikz}
\usepackage[unicode]{hyperref}
\usepackage{pgfplots,standalone}
\usepackage
	{
		% Дополнения Американского математического общества (AMS)
		amssymb,
		amsfonts,
		amsmath,
		amsthm,
		physics
		}
% \usepackage{lmodern}
\pgfplotsset{compat=newest} 
\usetikzlibrary{%
    decorations.pathreplacing,%
    decorations.pathmorphing,%
    patterns,%
    angles,%
    quotes,%
    calc, %
    3d, %
    backgrounds, %
    positioning%
}


% Стиль презентации

 \usetheme{Warsaw}
 \usefonttheme{professionalfonts}
 \usecolortheme{}
 % \usecolortheme{whale}

 

% \setbeamercolor{frametitle right}{fg=white,bg=Brown!85}
% \setbeamercolor{frametitle}{fg=white,bg=Brown!85}
% \setbeamercolor{frametitle right}{fg=white,bg=black!85} %
\setbeamercolor{frametitle}{fg=white,bg=black!85} % Цвет титульника

\setbeamertemplate{headline}{}
\setbeamertemplate{footline}{}
\let\Tiny=\tiny % решает проблему со шрифтами в TexLive
\setbeamertemplate
	{footline}{
		\color{black!40!white}
		\quad\hfill
		\insertframenumber/\inserttotalframenumber
		\hfill\vspace{1cm}\quad
	} 

\setbeamertemplate{navigation symbols}{}

\beamersetrightmargin{1cm} 
\beamersetleftmargin{1cm}

\setbeamertemplate{enumerate item}{
	\usebeamercolor[bg]{item projected}
	\raisebox{1pt}{\colorbox{bg}{\color{fg}\footnotesize\bf\insertenumlabel}}%
}
\setbeamercolor{item projected}{bg=black,fg=white}


\setbeamertemplate{itemize item}{%
	\usebeamercolor[bg]{item projected}%
	\raisebox{1pt}{{\color{bg}\footnotesize$\bf\square$}}%
}
\setbeamercolor{item projected}{bg=black,fg=white}
\setbeamercolor{title}{bg=black,fg=white}

\title[Измерение плотности плазмы]{Измерение плотности плазмы}

\author{%
	Виноградов И.Д. %
	Понур К.А. %
	Шиков А.П. %
}

\institute{Радиофизический факультет ННГУ, 430 группа}

\date{Нижний Новгород, 2018}

\begin{document}  

\section{Введение}
\subsection{Цели работы}
\begin{frame}[t]
	\frametitle{Цели работы}
	% \textbf{Цели}\\
	
		\vfill

		\begin{enumerate}
			\item Изучить принцип работы зонда с СВЧ-резонатором
			\item Измерить плотность и концентрацию плазмы в экспериментальном стенде "КРОТ"
		\end{enumerate}
		\vfill

\end{frame}
% \section{Теоретическая часть} % (fold)
% \label{sec:теоретическая_часть}

% \subsection{Понятие плазмы} % (fold)
% \label{sub:понятие_плазмы}
% % \begin{frame}
		
% % \end{frame}
% \subsection{Источник плазмы} % (fold)
% \label{sub:генерация_плазмы}
% % \begin{frame}
	
% % \end{frame}

\subsection{Зонд с СВЧ-резонатором} % (fold)

\begin{frame}
	\frametitle{Зонд с СВЧ-резонатором}
	\begin{equation}
		\omega_{res}=\left(\frac\pi{2l}\right)\frac c{\sqrt{\varepsilon}}
	\end{equation}
	В плазме: $\omega_{res}^2=\omega_{0res}^2+\omega_p^2$, 
	где $\omega_{0res}=\frac{\pi c}{2l}$

	Концентрация плазмы
	\begin{equation}
		N=\frac{m\omega^2_p}{4\pi e^2}
	\end{equation}
\end{frame}
% \section{Практическая часть} % (fold)
% \label{sec:практическая_часть}

% \label{sub:концентрация_плазмы}

% \subsection{Плотность плазмы} % (fold)
% \label{sub:плотность_плазмы}

% % subsection плотность_плазмы (end)

% % subsection концентрация_плазмы (end)
% % section практическая_часть (end)
% % subsection зонд_с_свч_резонатором (end)
% % subsection генерация_плазмы (end)
% % subsection понятие_плазмы (end)

% % section теоретическая_часть (end)


\end{document}