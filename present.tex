
%!TEX root = ../plasma.tex
\usepackage[english,russian]{babel}
% \usepackage[T2A,T1]{fontenc}
\usepackage[utf8]{inputenc}
% \usepackage{tikz}
\usepackage[unicode]{hyperref}
% \usepackage{pgfplots,standalone}
\usepackage{caption}
\usepackage[normalem]{ulem}
\usepackage
	{
		% Дополнения Американского математического общества (AMS)
		amssymb,
		amsfonts,
		amsmath,
		amsthm,
		physics
		}
% \usepackage{lmodern}
% \pgfplotsset{compat=newest} 
% \usetikzlibrary{%
%     decorations.pathreplacing,%
%     decorations.pathmorphing,%
%     patterns,%
%     angles,%
%     quotes,%
%     calc, %
%     3d, %
%     backgrounds, %
%     positioning%
% }


% Стиль презентации

 \usetheme{default}
 \usefonttheme{professionalfonts}
 \usecolortheme{}
 % \usecolortheme{whale}
% \let\oldframe\enumerate
% \renewcommand{\frame}{%
% \oldframe
% \let\olditemize\itemize
% \renewcommand\itemize{\olditemize\addtolength{\itemsep}{100pt}}%
% }
 

% \setbeamercolor{frametitle right}{fg=white,bg=Brown!85}
% \setbeamercolor{frametitle}{fg=white,bg=Brown!85}
%\setbeamercolor{frametitle right}{fg=white,bg=black!85} %
\setbeamercolor{frametitle}{fg=white,bg=black!85} % Цвет титульника
\setbeamercolor{item projected}{fg=white,bg=black!85} % Цвет титульника
\setbeamertemplate{blocks}[rounded][shadow=false] %стиль блоков
\setbeamertemplate{itemize item}{\color{black!85}$\bullet$}
\setbeamertemplate{headline}{}
\setbeamertemplate{footline}{} 
\setbeamertemplate{navigation symbols}{} % минус навигация
% \let\Tiny=\tiny % решает проблему со шрифтами в TexLive
%\setbeamertemplate
%	{footline}{
%		\color{black!40!white}
%		\quad\hfill
%		\insertframenumber/\inserttotalframenumber
%		\hfill\vspace{1cm}\quad
%	} 


\beamersetrightmargin{1cm} 
\beamersetleftmargin{1cm}

\setbeamertemplate{enumerate item}
{
	\usebeamercolor[bg]{item projected}
	\raisebox{1pt}{\colorbox{black!85}{\color{fg}\footnotesize\insertenumlabel}}%
}

% \setbeamertemplate{itemize item}{%
% 	\usebeamercolor[bg]{item projected}%
% 	\raisebox{3pt}{{\colorbox{black!85}\footnotesize$\bf$\bullet}}%
% }

\setbeamercolor{item projected}{bg=black,fg=white}
\setbeamercolor{title}{bg=black!85,fg=white}

\setbeamertemplate{frametitle}
{	
	\nointerlineskip
	\begin{beamercolorbox}[sep=15pt,ht=1.9em,wd=\paperwidth]{frametitle}
		% \vspace{}%
		\strut\insertframetitle\strut
		\vskip-1.8ex%	
	\end{beamercolorbox}
}



\begin{document}
\section{Актуальность работы} % (fold)
\subsection{Газовый разряд}
Первая работа по физике плазмы была выполнена Ленгмюром, Тонксом и их сотрудниками в 1920-х гг. Это исследование было вызвано
необходимостью разработать вакуумные электронные лампы, 
которые могли бы пропускать большие токи, а для этого их нужно
было наполнять ионизованным газом. Именно в этой работе
было открыто явление экранирования; В настоящее время мы сталкивается с газовым разрядом
в ртутных выпрямителях, водородных тиратронах, игнитронах,
разрядниках, сварочных дугах, неоновых лампах и лампах 
дневного света, в грозовых разрядах.


\subsection{Изучение космического окружения Земли.}

Непрерывный поток заряженных частиц, называемый солнечным ветром,
сталкивается с земной магнитосферой, которая деформируется под его действием и защищает нас от этого потока частиц.

Ионосфера, простирающаяся по высоте от 50 км до 10 земных радиусов, заполнена слабоионизированной плазмой, плотность которой изменяется с высотой. 

\subsection{Современная астрофизика}

Звезды и из атмосферы настолько горячи, что находятся в плазменном состоянии. Солнечное излучение обусловлено термоядерными реакциями, протекающими при высокой температуре. Солнечная корона представляет собой разреженную плазму. Межзвездная среда содержит ионизированный водород. 
Хотя звезды в галактиках не являются заряженными, они ведут себя подобно частицам в плазме. Поэтому для предсказания хода эволючии галактик применялась кинетическая теория плазмы. Радиоастрономия открыла многочисленные источники излучения, которые создаются плазмой. 

\subsection{МГД-преобразование энергии}

Для генерации электричества можно использовать МГД преобразование энергии плотной плазменной струи, движущейся поперек внешнего магнитного поля. Под действием силы Лоренца ионы движутся в одну сторону, а электроны в другую, что создаёт разность потенциалов, между двумя электродами. При этом с электродов можно снимать электрический ток, минуя неэффективный тепловой цикл.

Такой же принцип применяется в разработках ионных двигателей.

\subsection{Газовые лазеры}

Наиболее широко распространенным методом накачки казового лазера, т.е. перевода его в инвертированное состояние, которое может привести к усилению излучения, является применен
ие газового разряда.
\section{Зонд с СВЧ-резонатором} % (fold)
% \label{sec:зонд_с_свч_резонатором}
\subsection{Основы метода локальных измерений концентрации плазмы с использованием зонда с СВЧ-резонатором}
Идея, положенная в основу простого и удобного метода локальных измерений плотности плазмы, заключается в измерении \underline{собственной частоты}
миниатюрного резонатора, помещенного в плазму. Зонд представляет собой резонансную систему, собственная частота которой зависит от диэлектрической проницаемости среды $\epsilon$. По величине резонансной частоты однозначно восстанавливается значение плотности плазмы.

Для локальных измерений плотности плазмы используется простейший и наименьший по размерам резонатор, который является четверть-волновый отрезок двухпроводной линии, замкнутый на одном и разомкнутый на другом конце  $\hyperref[def:1]{\text{(четвертьволновый резонатор)}}$. Возбуждение и прием сигнала осуществляется при помощи двух передающих линийЮ оканчивающихся петлями магнитной связи.

Необходимым условием работы диагностики является требование, чтобы собственная частота резонатора с плазмой была значительно больше плазменной частоты . В этом случае ($\omega_{res}>>\omega_{pe}$, res--резонатор, pe-- плазменная частота) моды плазменных колебаний, которые могут возбуждаться в теплой плазме подавляются затуханием Ландау и поэтому не влияют на результаты измерений.

Малостью размеров зонда обеспечивается слабость возмущения, вносимого им в плазму.

В отличие от диагностических методов, связанных с использованием объемных резонаторов, позволяющих получить лишь интегральное значение плотности плазмы, предлагаемый метод позволяет (из-за малости размеров зонда по сравнению с характерными масштабами плазменного столба) определять локальное значение плотности внутри плазменного объема.

В сравнении с традиционно используемыми в плазменных экспериментах ленгмюровскими зондами, \textbf{результаты измерений с помощью резонансного СВЧ-зонда определяются только плотностью плазмы и не зависят от электронной температуры плазмы}.

Таким образом, диагностика плотности плазмы с помощью резонансного СВЧ-зонда позволяет измерять локальное, слабо возмущенное значение плотности плазмы.
% section зонд_с_свч_резонатором (end)
Нелинейные свойства СВЧ-резонатора, в том числе и гистерезисные явления, проявляются при больших амплитудах колебаний СВЧ-поля. В этом случае плазма вытесняется полем из всей области между проводами резонатора и резонансная кривая в районе своего максимума перестает зависеть от плазменной концентрации, практически полностью повторяя вакуумную. 
% section актуальность_работы (end)

\section{Экспериментальная установка КРОТ} % (fold)

Следует отметить, что при постановке космических исследований приходится иметь дело с трудными и дорогостоящими экспериментами. Это делает оправданным изучение космических эффектов в модельных экспериментах, проводимых на лабораторных установках, тем более что основные процессы, как в космической плазме, так и в лабораторной, при правильном выборе условий эксперимента подчиняются одним и тем же закономерностям. Кроме того, в лаборатории можно использовать весь арсенал современной диагностики плазмы и многократно воспроизводить исследуемое явление, целенаправленно варьировать условия его протекания. Возможность моделирования космических электромагнитных явлений основывается на законах подобия. Они указывают, как должны соотноситься между собой основные безразмерные физические величины в космическом объекте и его лабораторном аналоге. 

Объем, в котором проходит исследование плазменных процессов в нашем случае
представляет из себя вакуумную камеру, изготовленную из немагнитной нержавеющей стали диаметром 3 метра и длиной 10 метров. 
Предельный объем достигаемый в объеме камере, $P_{\text{ост}}\approx 5\cdot 10^{-6}$ торр. Откачка в камере осуществляется с помощью вакуумных насосов с производительностью 150 л/с.  Для откачки инертных газов используется пароструйные насосы с производительностью $2.5\cdot10^3$ л/сек. 

\subsection{Генерация плазмы} % (fold)
% \label{sub:генерация_плазмы}
В установке присутствует соленоид, который генерирует поле пробочной конфигурации, то есть создает магнитную ловушку для удержания плазмы.

В качестве источника плазмы используется высокочастотный автогенератор, нагруженный на индуктор, выполненный в виде двух витков разного диаметра, разнесенных в пространстве. Под действием ВЧ-поля происходит газовый разряд, инициирующий ионизирующую лавину. В результате образующаяся плазма экранирует индуктор и генерация прекращается.  

% subsection генерация_плазмы (end)

\subsection{Эксперимент} % (fold)
\subsubsection{Распад}

На слайде приведён график зависимости плотности плазмы от времени. Исходя из результатов измерения, можно сделать заключение, что  распад плазмы имеет два характерных времени. Это связано с тем, что характерное время спада концентрации обратно пропорционально электронной температуре плазмы.
Температура в ходе эксперимента меняется неравномерно.
При больших температурах остывание идёт быстрее, но через некоторое время оно заметно замедляется. Это связано с тем, что температура электронов приближается к температуре нейтрального газа. В результате получается некое подобие изотермы. 



\subsubsection{Радиальное распределение}
На практике считается, что  при отклонении концентраций плазмы около 50 \% плазму можно считать равномерно распределенной по радиальной оси. 
Но всё же концентрация непостоянна из-за особенностей генерации плазмы.
 
  

% subsection эксперимент (end)
\end{document}